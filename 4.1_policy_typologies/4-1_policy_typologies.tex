\documentclass{beamer}
\usetheme{Madrid}
\usecolortheme{dolphin}

\title{Policies and Policy Types}
\subtitle{POSC 315 - Politics and Policy Making in America}
\author{David P. Adams, Ph.D.}
\date{Week 4, Lecture 1}

\begin{document}

\frame{\titlepage}

\begin{frame}
\frametitle{Policies}
\begin{itemize}
    \item Some types involve more interest groups and publics than others
    \item Some engender more conflict than others
    \item Some are more visible than others
    \item Some can transform inattentive publics into attentive publics
\end{itemize}
\end{frame}

\begin{frame}
\frametitle{Policy Typologies}
\begin{enumerate}[A.]
    \item Help categorize things
    \item Help predict what sort of politics will accompany particular kinds of policies
    \item Categories aren't always perfect
    \item A policy can transform into different types over time
    \item A policy can fit into multiple types at the same time
\end{enumerate}
\end{frame}

\begin{frame}
\frametitle{Classic Policy Typologies}
\begin{enumerate}[A.]
    \item Distributive Policies
    \item Regulatory Policies
    \item Redistributive Policies
\end{enumerate}
\end{frame}

\begin{frame}
\frametitle{Distributive Policies}
\begin{itemize}
    \item Takes resources from a broad group and gives to a narrower group
    \item Can result from logrolling and backscratching (e.g., pork-barrel politics)
    \item Often results in "Interest Group Liberalism"
    \item Raises questions about equality vs. equity
\end{itemize}
\end{frame}

\begin{frame}
\frametitle{Regulatory Policies}
Two types:
\begin{enumerate}
    \item Competitive Regulatory Policy
    \begin{itemize}
        \item Limit provision of goods/services to designated deliverers
        \item E.g., licensed professionals, radio frequencies
        \item Low-visibility policies
    \end{itemize}
    \item Protective Regulatory Policy
    \begin{itemize}
        \item Protect public from negative effects of private activity
        \item E.g., unsafe products, pollution, tainted food
        \item High-visibility policies
    \end{itemize}
\end{enumerate}
\end{frame}

\begin{frame}
\frametitle{Redistributive Policies}
\begin{itemize}
    \item Take resources from one identifiable group, give to another
    \item Manipulate allocation of wealth, property, personal/civil rights
    \item Can work both ways:
    \begin{itemize}
        \item Most well off to least well off (e.g., new taxes)
        \item Least well off to most well off (e.g., tax cuts)
    \end{itemize}
    \item Not always about money (e.g., civil rights legislation)
    \item Highly visible and often controversial
\end{itemize}
\end{frame}

\begin{frame}
\frametitle{Alternative Policy Typologies}
\begin{enumerate}[A.]
    \item Cost--Benefit Analysis
    \item Substantive and Procedural Policies
    \item Material versus Symbolic Policies
    \item Liberal versus Conservative Policies
\end{enumerate}
\end{frame}

\begin{frame}
\frametitle{Cost--Benefit Analysis}
\begin{itemize}
    \item Concentrated or Diffuse
    \item Based on extent of focus on particular interests
    \item Costs and benefits can be socially constructed
\end{itemize}
\end{frame}

\begin{frame}
\frametitle{Substantive and Procedural Policies}
\begin{itemize}
    \item Substantive: What government does
    \item Procedural: How government pursues its goals
\end{itemize}
\end{frame}

\begin{frame}
\frametitle{Material vs Symbolic Policies}
\begin{itemize}
    \item Material: Doing something concrete
    \item Symbolic: Appealing to values
\end{itemize}
\end{frame}

\begin{frame}
\frametitle{Liberal vs Conservative Policies}
\begin{itemize}
    \item Easiest to generalize, least useful to analyze
    \item Liberals: Government can solve problems
    \item Conservatives: Government is often the problem
\end{itemize}
\end{frame}

\end{document}