\documentclass[11pt, aspectratio=169]{beamer}

\usetheme{metropolis}
\usecolortheme{seahorse}
\usepackage{graphicx}
\usepackage{fontspec}
\setmainfont{DejaVu Sans}


\title{Environmental and Energy Policy:\\ Theory and Practice}
\subtitle{Understanding Policy Development, Implementation \& Change}
\author{David Adams, Ph.D.}
\institute{Division of Politics, Administration, and Justice\\
    Cal State Fullerton}
\date{Fall 2024}

\begin{document}

\frame{\titlepage}

% Lecture Overview
\begin{frame}{Lecture Overview}
    \begin{itemize}
        \item \textbf{Lecture Objectives:}
            \begin{itemize}
                \item Analyze environmental and energy policies through a public policy lens
                \item Understand the role of policy in addressing environmental challenges
                \item Evaluate policy effectiveness and implementation strategies
                \item Apply policy theories to real-world environmental issues
            \end{itemize}
        \item \textbf{Key Themes:}
            \begin{itemize}
                \item Policy formulation and agenda setting
                \item Policy implementation and evaluation
                \item Stakeholder engagement and governance
                \item Equity and justice in environmental policy
            \end{itemize}
    \end{itemize}
\end{frame}

% Table of Contents
\begin{frame}{Table of Contents}
    \tableofcontents
\end{frame}

% Section 1: Introduction
\section{Introduction}

\begin{frame}{Why Study Environmental and Energy Policy in Public Policy?}
    \begin{itemize}
        \item \textbf{Policy Relevance:} Central to current public policy agendas
        \item \textbf{Complex Challenges:} Climate change, sustainability, resource management
        \item \textbf{Interdisciplinary Approach:} Integrating economic, social, and environmental considerations
        \item \textbf{Professional Application:} Careers in policy analysis, advocacy, public administration
    \end{itemize}
\end{frame}

\begin{frame}{The Role of Public Policy in Environmental Issues}
    \begin{itemize}
        \item \textbf{Government Intervention:} Addressing market failures and externalities
        \item \textbf{Policy Instruments:} Regulations, taxes, subsidies, information campaigns
        \item \textbf{Balancing Interests:} Economic development vs. environmental protection
        \item \textbf{International Dimensions:} Global coordination for transboundary issues
    \end{itemize}
\end{frame}

\begin{frame}{Key Questions in Environmental and Energy Policy}
    \begin{itemize}
        \item \textbf{How are environmental policies developed and implemented?}
        \item \textbf{What are the main challenges in addressing environmental issues?}
        \item \textbf{How can policy promote sustainability and equity?}
        \item \textbf{What are the future directions of environmental and energy policy?}
    \end{itemize}
\end{frame}

% Section 2: Theoretical Foundations
\section{Theoretical Foundations}

\begin{frame}{Understanding Environmental Policy}
    \begin{itemize}
        \item \textbf{Definition:} Public policies managing human impact on the environment
        \item \textbf{Policy Actors:} Governments, NGOs, businesses, the public
        \item \textbf{Policy Levels:} Local, state, national, international
        \item \textbf{Policy Processes:} Agenda setting, formulation, implementation, evaluation
    \end{itemize}
\end{frame}

\begin{frame}{Policy Process Theories}
    \begin{itemize}
        \item \textbf{Multiple Streams Framework:} Convergence of problems, policies, politics
        \item \textbf{Advocacy Coalition Framework:} Coalitions influencing policy change
        \item \textbf{Punctuated Equilibrium Theory:} Periods of stability and sudden change
        \item \textbf{Institutional Analysis and Development:} Role of institutions in policy
    \end{itemize}
\end{frame}

\begin{frame}{Multiple Streams Framework}
    \begin{itemize}
        \item \textbf{Problem Stream:} Recognizing issues requiring attention
        \item \textbf{Policy Stream:} Developing feasible solutions
        \item \textbf{Politics Stream:} Political climate and public mood
        \item \textbf{Policy Window:} Opportunity for policy change when streams align
    \end{itemize}
    \pause
    \textbf{Discussion:} Can you think of recent environmental issues where a policy window was open?
\end{frame}

\begin{frame}{Advocacy Coalition Framework}
    \begin{itemize}
        \item \textbf{Policy Subsystems:} Specific areas with various actors
        \item \textbf{Coalitions:} Groups sharing beliefs and coordinating actions
        \item \textbf{Policy Learning:} Changes in beliefs through experience
        \item \textbf{External Events:} Economic crises, disasters influencing change
    \end{itemize}
    \pause
    \textbf{Discussion:} How do advocacy coalitions shape environmental policy debates?
\end{frame}

\begin{frame}{Punctuated Equilibrium Theory}
    \begin{itemize}
        \item \textbf{Policy Stability:} Long periods of incremental change
        \item \textbf{Policy Shifts:} Rapid changes due to crises or new information
        \item \textbf{Feedback Loops:} Policy impacts influencing future decisions
        \item \textbf{Policy Entrepreneurs:} Actors driving change during punctuations
    \end{itemize}
    \pause
    \textbf{Discussion:} What are examples of policy punctuations in environmental policy?
\end{frame}

\begin{frame}{Institutional Analysis and Development}
    \begin{itemize}
        \item \textbf{Institutions:} Formal and informal rules shaping behavior
        \item \textbf{Path Dependency:} Historical legacies influencing current policy
        \item \textbf{Transaction Costs:} Costs of policy change and coordination
        \item \textbf{Policy Feedback:} Policies affecting institutions and vice versa
    \end{itemize}
    \pause
    \textbf{Discussion:} How do institutions influence environmental policy outcomes?
\end{frame}

% Section 3: Historical Context
\section{Historical Context}

\begin{frame}{Evolution of Environmental Policy}
    \begin{itemize}
        \item \textbf{Early Conservation:} Preservation of natural resources
        \item \textbf{Modern Movement:} Rise in the 1960s-1970s environmental awareness
        \item \textbf{Legislative Milestones:} Clean Air Act, NEPA
        \item \textbf{Policy Shifts:} From command-and-control to market-based approaches
    \end{itemize}
\end{frame}

\begin{frame}{Key Environmental Legislation}
    \begin{itemize}
        \item \textbf{NEPA (1969):} Environmental impact assessments
        \item \textbf{Clean Air Act (1970):} Air quality standards
        \item \textbf{Clean Water Act (1972):} Water pollution regulation
        \item \textbf{Endangered Species Act (1973):} Species and habitat protection
    \end{itemize}
    \pause
    \textbf{Discussion:} How have these laws shaped current environmental policy?
\end{frame}

\begin{frame}{International Environmental Agreements}
    \begin{itemize}
        \item \textbf{Montreal Protocol (1987):} Ozone layer protection
        \item \textbf{Kyoto Protocol (1997):} Greenhouse gas emissions
        \item \textbf{Paris Agreement (2015):} Climate change mitigation
        \item \textbf{Sustainable Development Goals (2015):} Global development targets
    \end{itemize}
    \pause
    \textbf{Discussion:} How do international agreements influence national policy?
\end{frame}

\begin{frame}{Policy Shifts in Energy}
    \begin{itemize}
        \item \textbf{Traditional Energy Sources:} Coal, oil, natural gas
        \item \textbf{Renewable Energy Transition:} Solar, wind, hydro
        \item \textbf{Energy Efficiency:} Reducing demand and emissions
        \item \textbf{Technological Innovation:} Smart grids, electric vehicles
    \end{itemize}
    \pause
    \textbf{Discussion:} How have energy policies evolved to address sustainability?
\end{frame}

% Section 4: Current Environmental Challenges
\section{Current Environmental Challenges}

\begin{frame}{Climate Change}
    \begin{itemize}
        \item \textbf{Global Warming:} Rising temperatures, greenhouse gases
        \item \textbf{Impacts:} Sea-level rise, extreme weather
        \item \textbf{Policy Responses:} Paris Agreement, national strategies
    \end{itemize}
    \pause
    \textbf{Discussion:} What are the policy challenges in addressing climate change?
\end{frame}

\begin{frame}{Environmental Justice}
    \begin{itemize}
        \item \textbf{Definition:} Fair treatment in environmental policies
        \item \textbf{Issues:} Burdens on marginalized communities
        \item \textbf{Policy Considerations:} Equity in design and implementation
    \end{itemize}
    \pause
    \textbf{Discussion:} How can policy address environmental justice concerns?
\end{frame}

\begin{frame}{Resource Management}
    \begin{itemize}
        \item \textbf{Water Scarcity:} Droughts, pollution
        \item \textbf{Land Use:} Urban sprawl, deforestation
        \item \textbf{Policy Tools:} Conservation, land planning
    \end{itemize}
    \pause
    \textbf{Discussion:} How can policy balance resource use and conservation?
\end{frame}

\begin{frame}{Biodiversity Loss}
    \begin{itemize}
        \item \textbf{Habitat Destruction:} Development, climate change
        \item \textbf{Species Extinction:} Loss of biodiversity
        \item \textbf{Policy Responses:} Protected areas, conservation efforts
    \end{itemize}
    \pause
    \textbf{Discussion:} What are the policy implications of biodiversity loss?
\end{frame}

% Section 5: Energy Policy Landscape
\section{Energy Policy Landscape}

\begin{frame}{Traditional Energy Sources}
    \begin{itemize}
        \item \textbf{Fossil Fuels:} Coal, oil, natural gas
        \item \textbf{Economic Importance:} Jobs, energy security
        \item \textbf{Environmental Impacts:} Pollution, emissions
    \end{itemize}
    \pause
    \textbf{Discussion:} What are the policy implications of reliance on fossil fuels?
\end{frame}

\begin{frame}{Renewable Energy Transition}
    \begin{itemize}
        \item \textbf{Renewable Sources:} Solar, wind, hydro
        \item \textbf{Policy Support:} Incentives, subsidies
        \item \textbf{Challenges:} Costs, infrastructure
    \end{itemize}
    \pause
    \textbf{Discussion:} How can policy accelerate renewable energy adoption?
\end{frame}

\begin{frame}{Energy Efficiency and Conservation}
    \begin{itemize}
        \item \textbf{Importance:} Reducing demand and emissions
        \item \textbf{Policy Measures:} Standards, incentives
        \item \textbf{Behavioral Aspects:} Influencing consumer habits
    \end{itemize}
    \pause
    \textbf{Discussion:} What policies promote energy efficiency effectively?
\end{frame}

\begin{frame}{Technological Innovation in Energy}
    \begin{itemize}
        \item \textbf{Smart Grids:} Efficient energy distribution
        \item \textbf{Electric Vehicles:} Reducing transportation emissions
        \item \textbf{Research and Development:} Support for clean technologies
    \end{itemize}
    \pause
    \textbf{Discussion:} How can policy foster innovation in the energy sector?
\end{frame}

% Section 6: Policy Implementation and Enforcement
\section{Policy Implementation and Enforcement}

\begin{frame}{Policy Implementation Strategies}
    \begin{itemize}
        \item \textbf{Regulatory Approaches:} Command-and-control
        \item \textbf{Market-Based Instruments:} Taxes, cap-and-trade
        \item \textbf{Voluntary Programs:} Certifications, partnerships
    \end{itemize}
    \pause
    \textbf{Discussion:} Advantages and disadvantages of these instruments?
\end{frame}

\begin{frame}{Challenges in Implementation}
    \begin{itemize}
        \item \textbf{Administrative Capacity:} Resources, expertise
        \item \textbf{Compliance:} Monitoring, enforcement
        \item \textbf{Political Opposition:} Stakeholder resistance
        \item \textbf{Coordination:} Across government levels
    \end{itemize}
    \pause
    \textbf{Discussion:} How can policy design address these challenges?
\end{frame}

\begin{frame}{Environmental Justice in Implementation}
    \begin{itemize}
        \item \textbf{Inclusive Processes:} Community engagement
        \item \textbf{Equitable Outcomes:} Fair distribution of benefits and burdens
        \item \textbf{Policy Tools:} Impact assessments, agreements
    \end{itemize}
    \pause
    \textbf{Discussion:} Role of public participation in environmental justice?
\end{frame}

\begin{frame}{Policy Evaluation and Enforcement}
    \begin{itemize}
        \item \textbf{Evaluation Criteria:} Effectiveness, efficiency, equity
        \item \textbf{Data and Metrics:} Measuring policy outcomes
        \item \textbf{Enforcement Mechanisms:} Penalties, incentives
    \end{itemize}
    \pause
    \textbf{Discussion:} How can policy evaluation improve environmental outcomes?
\end{frame}

% Section 7: Future Policy Directions
\section{Future Policy Directions}

\begin{frame}{Integrating Climate Policy}
    \begin{itemize}
        \item \textbf{National Strategies:} Emission targets, clean energy
        \item \textbf{International Cooperation:} Global agreements
        \item \textbf{Innovation:} Research and development support
    \end{itemize}
    \pause
    \textbf{Discussion:} What policies are needed to meet climate goals?
\end{frame}

\begin{frame}{Technological Innovation and Policy}
    \begin{itemize}
        \item \textbf{Clean Technologies:} Energy, transportation
        \item \textbf{Policy Incentives:} R\&D funding, tax credits
        \item \textbf{Regulatory Frameworks:} Standards, patents
    \end{itemize}
    \pause
    \textbf{Discussion:} How can policy support technological change?
\end{frame}

\begin{frame}{Equity and Justice in Policy}
    \begin{itemize}
        \item \textbf{Environmental Justice:} Fair treatment in policy
        \item \textbf{Social Equity:} Addressing disparities
        \item \textbf{Policy Design:} Inclusive processes, equitable outcomes
    \end{itemize}
    \pause
    \textbf{Discussion:} How can policy promote equity and justice?
\end{frame}

\begin{frame}{Governance and Stakeholder Engagement}
    \begin{itemize}
        \item \textbf{Collaborative Governance:} Inclusive decision-making
        \item \textbf{Transparency:} Open data, communication
        \item \textbf{Adaptive Management:} Flexibility in policy design
        \item \textbf{Public Participation:} Engaging diverse stakeholders
        \item \textbf{Policy Networks:} Building coalitions for change
        \item \textbf{Accountability:} Monitoring and evaluation
        \item \textbf{Resilience:} Preparing for future challenges
        \item \textbf{Innovation:} Encouraging new ideas and approaches
        \item \textbf{Learning:} Continuous improvement and adaptation
        \item \textbf{Sustainability:} Balancing economic, social, and environmental goals
    \end{itemize}
    \pause
    \textbf{Discussion:} How can governance improve policy outcomes?
\end{frame}


% Section 8: Case Study - My Research

\section{Case Studies: My Research}

\begin{frame}{Research Focus: Water Policy and Collaborative Governance}
    \begin{itemize}
        \item \textbf{Goal:} Explore how collaboration among stakeholders affects water quality outcomes in watersheds.
        \item \textbf{Key Themes:}
            \begin{itemize}
                \item Stakeholder Involvement: Importance of diverse interests in policy decision-making.
                \item Trust and Transparency: Building trust through open, collaborative processes.
                \item Adaptive Management: Using data and stakeholder input to adjust policies over time.
            \end{itemize}
        \item \textbf{Related Theory:} \textit{Collaborative Governance Framework}
            \begin{itemize}
                \item Highlights how collaboration improves policy implementation and compliance.
                \item Applied to manage complex environmental systems, especially in water policy.
            \end{itemize}
    \end{itemize}
\end{frame}

\begin{frame}{Research Focus: Environmental Justice in Oil and Gas Extraction}
    \begin{itemize}
        \item \textbf{Goal:} Examine the impact of oil spills and extraction activities on vulnerable communities.
        \item \textbf{Key Findings:}
            \begin{itemize}
                \item Disproportionate Spill Locations: Spills often occur near low-income or marginalized communities.
                \item Delays in Spill Reporting: Data shows longer reporting times in certain areas, raising equity concerns.
            \end{itemize}
        \item \textbf{Related Theory:} \textit{Advocacy Coalition Framework}
            \begin{itemize}
                \item Environmental coalitions vs. industry coalitions in influencing policy.
                \item Insights into how power and information asymmetry impact policy enforcement.
            \end{itemize}
        \item \textbf{Real-World Application:} Supports policy reforms for equitable spill response and prevention strategies.
    \end{itemize}
\end{frame}

\begin{frame}{Research Focus: Energy Transitions and Sustainability}
    \begin{itemize}
        \item \textbf{Goal:} Investigate the socio-economic impacts of transitioning from fossil fuels to renewable energy.
        \item \textbf{Key Issues:}
            \begin{itemize}
                \item Economic Impacts: Job creation in renewable sectors vs. job loss in traditional energy.
                \item Social Equity: Ensuring fair access to new energy opportunities across communities.
                \item Environmental Sustainability: Reducing carbon emissions and promoting long-term ecological health.
            \end{itemize}
        \item \textbf{Related Theory:} \textit{Punctuated Equilibrium Theory}
            \begin{itemize}
                \item Explains rapid policy shifts in response to crises or major technological advances.
                \item Relevant to the recent push for renewable energy due to climate pressures.
            \end{itemize}
        \item \textbf{Real-World Application:} Supports policies that balance economic, environmental, and social priorities.
    \end{itemize}
\end{frame}

\begin{frame}{Research Focus: Public Policy Outcomes and Equity}
    \begin{itemize}
        \item \textbf{Goal:} Assess the effectiveness of policies in achieving fair outcomes, especially for disadvantaged groups.
        \item \textbf{Key Concepts:}
            \begin{itemize}
                \item Policy Effectiveness: Measuring how well policies meet stated objectives.
                \item Accountability and Transparency: Essential for fair implementation and public trust.
                \item Equity in Outcomes: Ensuring policies don’t disproportionately impact vulnerable groups.
            \end{itemize}
        \item \textbf{Related Theory:} \textit{Policy Evaluation and Equity Frameworks}
            \begin{itemize}
                \item Evaluates policies based on criteria like efficiency, effectiveness, and equity.
                \item Focus on minimizing unintended consequences for disadvantaged groups.
            \end{itemize}
        \item \textbf{Real-World Impact:} Informs policy adjustments to ensure fairness in public administration practices.
    \end{itemize}
\end{frame}

\begin{frame}{Research Focus: Cross-Cutting Themes and Future Directions}
    \begin{itemize}
        \item \textbf{Intersections of Policy Areas:}
            \begin{itemize}
                \item \textbf{Environmental Justice and Energy Policy:} Ensuring that energy transitions do not disproportionately impact marginalized groups.
                \item \textbf{Collaborative Governance and Policy Effectiveness:} Demonstrating how stakeholder engagement improves outcomes in both water policy and energy.
            \end{itemize}
        \item \textbf{Emerging Areas:}
            \begin{itemize}
                \item \textbf{Data-Driven Policy:} Using data analytics to improve transparency and accountability in policy implementation.
                \item \textbf{Adaptive Policy Frameworks:} Flexible policies that respond to changing environmental and social conditions.
            \end{itemize}
        \item \textbf{Impact on Public Administration:} Provides a foundation for policies that are resilient, equitable, and responsive to community needs.
    \end{itemize}
\end{frame}

% Section 9: Conclusion
\section{Conclusion}

\begin{frame}{Key Takeaways}
    \begin{itemize}
        \item \textbf{Interconnectedness:} Environmental and energy policies are linked
        \item \textbf{Holistic Approaches:} Need for integrated solutions
        \item \textbf{Active Participation:} Role of stakeholders in shaping policy
        \item \textbf{Continuous Learning:} Adapting to new challenges and information
    \end{itemize}
    \pause
    \begin{itemize}
    \item \textbf{Question:} What role can you play in advancing these policy goals, either as a citizen or as a public servant?
    \end{itemize}
    \pause
    \begin{itemize}
    \item \textbf{Question:} What are your main takeaways from today, and what questions do you still have?
    \end{itemize} 
\end{frame}


\end{document}
