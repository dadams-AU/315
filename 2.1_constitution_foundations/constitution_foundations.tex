\documentclass[12pt]{beamer}
\usetheme{metropolis}
\usepackage{booktabs}
\usepackage{tabularx}
\usepackage{calc}
\usepackage{tikz}
\usepackage{hyperref}

% Setup for faculty images
\newlength{\imageheight}
\setlength{\imageheight}{3.5cm}

% Define CSUF brand colors
\definecolor{titanblue}{HTML}{00244E}
\definecolor{mediumblue}{HTML}{0F3F8C}
\definecolor{skyblue}{HTML}{EBFBFF}
\definecolor{titanorange}{HTML}{FF7900}
\definecolor{titangray}{HTML}{F5F5F5}
\definecolor{titantext}{HTML}{222222}

% Customize metropolis theme colors
\setbeamercolor{normal text}{fg=titantext, bg=white}
\setbeamercolor{alerted text}{fg=titanorange}
\setbeamercolor{example text}{fg=mediumblue}

% Title page colors
\setbeamercolor{title}{fg=titanblue, bg=white}
\setbeamercolor{subtitle}{fg=mediumblue, bg=white}
\setbeamercolor{institute}{fg=titanorange, bg=white}
\setbeamercolor{date}{fg=titanblue, bg=white}

% Frame title colors
\setbeamercolor{frametitle}{fg=white, bg=titanblue}
\setbeamercolor{framesubtitle}{fg=mediumblue, bg=white}

% Block environment colors
\setbeamercolor{block title}{fg=white, bg=titanblue}
\setbeamercolor{block body}{fg=titantext, bg=skyblue!10}

% Item colors
\setbeamercolor{itemize item}{fg=titanorange}
\setbeamercolor{itemize subitem}{fg=mediumblue}
\setbeamercolor{itemize subsubitem}{fg=titanblue}

% Footer and header colors
\setbeamercolor{footer}{fg=titantext}
\setbeamercolor{header}{fg=titanblue}

% Customize fonts
\setbeamerfont{title}{size=\Large, series=\bfseries}
\setbeamerfont{subtitle}{size=\large}
\setbeamerfont{frametitle}{size=\large, series=\bfseries}
\setbeamerfont{itemize/enumerate body}{size=\small}
\setbeamerfont{itemize/enumerate subbody}{size=\footnotesize}

% Simple title page template
\defbeamertemplate*{title page}{customized}[1][]
{
    \vspace{1cm}
    {\usebeamerfont{title}\usebeamercolor[fg]{title}\inserttitle\par}
    \vspace{0.5cm}
    {\usebeamerfont{subtitle}\usebeamercolor[fg]{subtitle}\insertsubtitle\par}
    \vspace{0.5cm}
    {\usebeamerfont{date}\usebeamercolor[fg]{date}\insertdate\par}
    \vfill
    {\insertinstitute\par}
}

% Add progress bar
\makeatletter
\setbeamertemplate{headline}{%
  \begin{beamercolorbox}[wd=\paperwidth,ht=0.4cm,dp=0cm]{titanblue}%
    \begin{tikzpicture}
      \fill[titanorange] (0,0) rectangle (\the\paperwidth*\insertframenumber/\inserttotalframenumber,0.4cm);
    \end{tikzpicture}%
  \end{beamercolorbox}%
}
\makeatother

\begin{document}

\title{The Constitutional Order and Foundations of American Governance}
\subtitle{POSC 315: Introduction to Public Policy}
\author{Dr. David P. Adams}
\date{Week 2, Lecture 1}

\begin{frame}
    \titlepage
\end{frame}

\begin{frame}{Introduction}
    \begin{itemize}
        \item Justice Scalia's death sparked debates over judicial appointments.
        \item Political gridlock highlights the importance of constitutional interpretation.
        \item The evolving role of policy analysis in navigating governance challenges.
        \item \textbf{Example:} Current challenges in Supreme Court confirmations.
    \end{itemize}
\end{frame}

\begin{frame}{Learning Objectives}
    By the end of this lecture, you will be able to:
    \begin{itemize}
        \item Define and identify key institutions in American governance.
        \item Explain the role of the Constitution in policymaking.
        \item Analyze the impact of separation of powers on policy development.
        \item Apply constitutional principles to current policy issues.
        \item Discuss the concept of federalism and its implications for policy.
    \end{itemize}
\end{frame}

\section{Institutions}
\begin{frame}{Institutions}
    \begin{block}{Definition}
        Rules that govern interactions and transactions.
    \end{block}
    \begin{itemize}
        \item \textbf{Examples:}
        \begin{itemize}
            \item Family, Business, Religion.
            \item Hospitals, Schools, Communities.
            \item Capitalism, Marriage, Voting.
            \item Education, Legislatures.
        \end{itemize}
    \end{itemize}
\end{frame}

\begin{frame}{Importance of Institutions in Public Policy}
    \begin{itemize}
        \item Shape behavior and decision-making.
        \item Provide structure for policy implementation.
        \item Influence policy outcomes.
        \item Create paths of dependency in policy evolution.
        \item \textbf{Example:} Voting systems shaping electoral outcomes.
    \end{itemize}
\end{frame}

\begin{frame}{Discussion Prompt - What Institutions Matter?}
    \begin{block}{Group Discussion}
        \begin{itemize}
            \item Which institutions do you interact with most in your daily life?
            \item How do they influence your decisions?
        \end{itemize}
    \end{block}
    \begin{itemize}
        \item Small group discussion (5 minutes).
        \item Share key insights with the class.
    \end{itemize}
\end{frame}

\section{The Constitutional Order}
\begin{frame}{The Constitution and Policymaking}
    \begin{itemize}
        \item \textbf{Living Document:} Adaptable through amendments and interpretation.
        \item \textbf{Purposeful Vagueness:} Encourages flexibility.
        \item \textbf{Elasticity:} Supports growth and responsiveness.
        \item \textbf{Example:} Evolution of the Commerce Clause in regulating markets.
    \end{itemize}
\end{frame}

\begin{frame}{Key Constitutional Provisions}
    \begin{itemize}
        \item \textbf{Article I, Section 8:}
        \begin{itemize}
            \item Commerce Clause.
            \item Elastic Clause.
        \end{itemize}
        \item \textbf{Amendment 14:}
        \begin{itemize}
            \item Due Process.
            \item Equal Protection.
        \end{itemize}
        \item \textbf{Amendment 10:}
        \begin{itemize}
            \item Federalism Structure.
        \end{itemize}
    \end{itemize}
\end{frame}

\begin{frame}{Constitutional Interpretation}
    \begin{columns}
        \column{0.5\textwidth}
        \textbf{Interpretative Approaches:}
        \begin{itemize}
            \item Originalism vs. Living Constitution.
            \item Strict vs. Broad interpretation.
        \end{itemize}
        
        \column{0.5\textwidth}
        \textbf{Key Concepts:}
        \begin{itemize}
            \item Role of precedent.
            \item Impact on policy.
            \item Judicial review.
        \end{itemize}
    \end{columns}
    \textbf{Example:} Landmark cases like Brown v. Board of Education.
\end{frame}

\section{Separation of Powers}
\begin{frame}{Three Branches of Government}
    \begin{enumerate}
        \item \textbf{Legislative (Article I):} Makes laws.
        \item \textbf{Executive (Article II):} Enforces laws.
        \item \textbf{Judicial (Article III):} Interprets laws.
    \end{enumerate}
\end{frame}

\begin{frame}{Role of Checks and Balances}
    \begin{itemize}
        \item Prevents concentration of power.
        \item Contributes to policymaking complexity.
        \item \textbf{Examples:}
        \begin{itemize}
            \item Veto power.
            \item Judicial review.
            \item Advice and consent.
        \end{itemize}
    \end{itemize}
\end{frame}

\section{Policy Actors}
\begin{frame}{Official Policy Actors}
    \begin{itemize}
        \item \textbf{Legislative Branch:}
        \begin{itemize}
            \item Drafting and passing laws.
            \item Committees and subcommittees.
        \end{itemize}
        \item \textbf{Executive Branch:}
        \begin{itemize}
            \item Proposing legislation.
            \item Executive orders.
        \end{itemize}
        \item \textbf{Judicial Branch:}
        \begin{itemize}
            \item Interpreting laws.
            \item Judicial review.
        \end{itemize}
    \end{itemize}
\end{frame}

\begin{frame}{Unofficial Policy Actors}
    \begin{itemize}
        \item \textbf{Interest Groups:}
        \begin{itemize}
            \item Lobbying and advocacy.
            \item Campaign contributions.
        \end{itemize}
        \item \textbf{Media:}
        \begin{itemize}
            \item Agenda-setting.
            \item Framing issues.
        \end{itemize}
        \item \textbf{Think Tanks:}
        \begin{itemize}
            \item Policy research.
            \item Public opinion influence.
        \end{itemize}
    \end{itemize}
    \textbf{Activity:} Investigate lobbying practices via \href{https://wethevoters.org}{We the Voters} series.
\end{frame}

\section{Federalism}
\begin{frame}{Federalism Overview}
    \begin{block}{Definition}
        Division of power between national and state governments.
    \end{block}
    \begin{itemize}
        \item \textbf{Constitutional Basis:} 10th Amendment.
        \item \textbf{Types of Powers:}
        \begin{itemize}
            \item Enumerated powers (federal).
            \item Reserved powers (states).
            \item Concurrent powers (shared).
        \end{itemize}
    \end{itemize}
\end{frame}

\begin{frame}{Improving Policy Capacity}
    \begin{itemize}
        \item Invest in public education and engagement initiatives.
        \item Encourage bipartisan cooperation to reduce gridlock.
        \item Strengthen state policy capacity through federal support.
    \end{itemize}
    \textbf{Activity:} Analyze public trust trends via \href{http://www.pewresearch.org}{Pew Research Center}.
\end{frame}

\begin{frame}{For Next Time}

\end{frame}

\end{document}
